\documentclass[12pt,twoside=false,a4paper,parskip]{scrbook}
\usepackage[utf8]{inputenc}
\usepackage{csquotes}
%\usepackage[ngerman]{babel}
\usepackage[english]{babel}
\usepackage{floatflt}
\usepackage{subfigure}
\usepackage[pdftex]{graphicx}
\usepackage[hidelinks]{hyperref}
\usepackage{color}
\usepackage{amssymb}
\usepackage{textcomp}
\usepackage{nicefrac}
\usepackage{scrhack}
\usepackage{pdfpages}
\usepackage{float}
\usepackage{pdflscape}
\usepackage{subfigure}
\usepackage{pdfpages}
\usepackage[verbose]{placeins}
\usepackage[markcase=ignoreuppercase,headsepline,plainfootsepline]{scrlayer-scrpage}
\usepackage{listings}
\usepackage{xcolor}
\usepackage{color}
\usepackage{caption}
\usepackage{subfigure}
\usepackage{epstopdf}
\usepackage{longtable}
\usepackage{setspace}
\usepackage{booktabs}
\usepackage[style=numeric,backend=biber,sorting=none]{biblatex}
\bibliography{references}
\usepackage[htt]{hyphenat}

%%%%%%%%%%%%%%%%%%%
%% definitions
%%%%%%%%%%%%%%%%%%%
\def\BaAuthor{Fabian Frank Werner}
\def\BaAuthorStudyProgram{E-Commerce}
\def\BaType{Bachelor's Thesis}
\def\BaTitle{The Impact of AI Transparency\\on Advertising Credibility}
\def\BaSupervisorOne{Prof.\ Dr.\ Karsten Huffstadt}
\def\BaSupervisorTwo{Prof.\ Dr.\ habil.\ Nicholas Müller}
\def\BaDeadline{\today}

\def\iswithfullname{1} % (un)comment this to create (non-)anonymous version
\ifdefined\iswithfullname
  \def\ShowBaAuthor{\BaAuthor}
\else
  \def\ShowBaAuthor{N.~N.}
\fi

\hypersetup{
pdfauthor={\ShowBaAuthor},
pdftitle={\BaTitle},
pdfsubject={Subject},
pdfkeywords={Keywords}
}

%%%%%%%%%%%%%%%%%%%
%% configs to include
%%%%%%%%%%%%%%%%%%%
\colorlet{punct}{red!60!black}
\definecolor{background}{HTML}{EEEEEE}
\definecolor{delim}{RGB}{20,105,176}
\colorlet{numb}{magenta!60!black}

\definecolor{gray}{rgb}{0.4,0.4,0.4}
\definecolor{darkblue}{rgb}{0.0,0.0,0.6}
\definecolor{cyan}{rgb}{0.0,0.6,0.6}

\definecolor{pblue}{rgb}{0.13,0.13,1}
\definecolor{pgreen}{rgb}{0,0.5,0}
\definecolor{pred}{rgb}{0.9,0,0}
\definecolor{pgrey}{rgb}{0.46,0.45,0.48}

\lstset{
  basicstyle=\ttfamily,
  columns=fullflexible,
  showstringspaces=false,
  commentstyle=\color{gray}\upshape
  linewidth=\textwidth
}

\lstdefinelanguage{json}{
    basicstyle=\normalfont\ttfamily,
    numbers=left,
    numberstyle=\scriptsize,
    stepnumber=1,
    numbersep=8pt,
    showstringspaces=false,
    breaklines=true,
    backgroundcolor=\color{background},
    literate=
     *{0}{{{\color{numb}0}}}{1}
      {1}{{{\color{numb}1}}}{1}
      {2}{{{\color{numb}2}}}{1}
      {3}{{{\color{numb}3}}}{1}
      {4}{{{\color{numb}4}}}{1}
      {5}{{{\color{numb}5}}}{1}
      {6}{{{\color{numb}6}}}{1}
      {7}{{{\color{numb}7}}}{1}
      {8}{{{\color{numb}8}}}{1}
      {9}{{{\color{numb}9}}}{1}
      {:}{{{\color{punct}{:}}}}{1}
      {,}{{{\color{punct}{,}}}}{1}
      {\{}{{{\color{delim}{\{}}}}{1}
      {\}}{{{\color{delim}{\}}}}}{1}
      {[}{{{\color{delim}{[}}}}{1}
      {]}{{{\color{delim}{]}}}}{1},
}

\lstset{language=xml,
  morestring=[b]",
  morestring=[s]{>}{<},
  morecomment=[s]{<?}{?>},
  stringstyle=\color{black},
  numbers=left,
  numberstyle=\scriptsize,
  stepnumber=1,
  numbersep=8pt,
  identifierstyle=\color{darkblue},
  keywordstyle=\color{cyan},
  backgroundcolor=\color{background},
  morekeywords={xmlns,version,type} % list your attributes here
}

\lstset{language=Java,
  showspaces=false,
  showtabs=false,
  tabsize=4,
  breaklines=true,
  keepspaces=true,
  numbers=left,
  numberstyle=\scriptsize,
  stepnumber=1,
  numbersep=8pt,
  showstringspaces=false,
  breakatwhitespace=true,
  commentstyle=\color{pgreen},
  keywordstyle=\color{pblue},
  stringstyle=\color{pred},
  basicstyle=\ttfamily,
  backgroundcolor=\color{background},
%  moredelim=[il][\textcolor{pgrey}]{$$},
%  moredelim=[is][\textcolor{pgrey}]{\%\%}{\%\%}
}

\newcommand*{\forcetwosidetitle}[1][1]{%
 \begingroup
   \cleardoubleoddpage
   \KOMAoptions{titlepage=true} % useful e.g. for scrartcl
   \csname @twosidetrue\endcsname
   \maketitle[{#1}]
 \endgroup
}

\begin{document}

%%%%%%%%%%%%%%%%%%%
%% Title Page
%%%%%%%%%%%%%%%%%%%

\frontmatter
\titlehead{\centering
  {Technical University of Applied Sciences Würzburg-Schweinfurt\\Faculty of Computer Science and Business Information Systems}}
\subject{\BaType}
\title{\BaTitle\\[15mm]}
\subtitle{\normalsize{Submitted to the Technical University of Applied Sciences Würzburg-Schweinfurt at the Faculty of Computer Science and Business Information Systems\\for the completion of the degree program in \BaAuthorStudyProgram.}}
\author{\ShowBaAuthor}
\date{\normalsize{Submitted on: \BaDeadline}}
\publishers{
  \normalsize{Primary Supervisor: \BaSupervisorOne}\\
  \normalsize{Secondary Supervisor: \BaSupervisorTwo}\\
}

\lowertitleback{
\centering\includegraphics[width=4cm]{barcode_default}

}
\forcetwosidetitle

%%%%%%%%%%%%%%%%%%%
%% Abstract
%%%%%%%%%%%%%%%%%%%

\chapter*{Abstract}

TODO

%\chapter*{Zusammenfassung}

%TODO

%%%%%%%%%%%%%%%%%%%
%% Acknowledgements
%%%%%%%%%%%%%%%%%%%

% \chapter*{Acknowledgements}

%%%%%%%%%%%%%%%%%%%
%% Table of Contents
%%%%%%%%%%%%%%%%%%%

\renewcommand{\contentsname}{Table of Contents}
\tableofcontents

%%%%%%%%%%%%%%%%%%%
%% List of Figures
%%%%%%%%%%%%%%%%%%%

\listoffigures
\addcontentsline{toc}{chapter}{List of Figures}

%%%%%%%%%%%%%%%%%%%
%% List of Tables
%%%%%%%%%%%%%%%%%%%

\listoftables
\addcontentsline{toc}{chapter}{List of Tables}

%%%%%%%%%%%%%%%%%%%
%% List of Algorithms, List of Abbreviations, List of Symbols
%%%%%%%%%%%%%%%%%%%

%%%%%%%%%%%%%%%%%%%
%% Contents
%%%%%%%%%%%%%%%%%%%

\mainmatter

\chapter{Introduction}\label{ch:introduction} % 4-5 Pages



\section{Background and Motivation}\label{sec:introduction_background_and_motivation}

\section{Problem Statement and Research Gap}\label{sec:introduction_problem_statement_and_research_gap}

\section{Research Question and Objectives}\label{sec:introduction_research_question_and_objectives}

\section{Thesis Outline}\label{sec:introduction_thesis_outline}



\chapter{Theoretical Foundations and Hypothesis Development}\label{ch:theoretical_foundations_and_hypothesis_development} % 14-18 Pages

The following chapter establishes the theoretical foundation for the present thesis and summarizes the current state of research. The objective is to develop a comprehensive understanding of the central constructs, critically review existing knowledge, and identify the research gaps that necessitate this study.

First, the core concepts of AI are defined within the context of digital advertising (Section \ref{sec:theoretical_foundations_and_hypothesis_development_artificial_intelligence_in_digital_advertising}). Following this, a detailed conceptualization of perceived advertising credibility, the central dependent variable of this study, is provided (Section \ref{sec:theoretical_foundations_and_hypothesis_development_conceptualizing_perceived_advertising_credibility}). Subsequently, the independent variable—AI transparency, disclosure, and labeling—is examined, along with current findings on consumer responses to such disclosures (Section \ref{sec:theoretical_foundations_and_hypothesis_development_ai_transparency_disclosure_and_labeling}).

A synthesis of current research (Section \ref{sec:theoretical_foundations_and_hypothesis_development_synthesis_of_current_research}) will then consolidate relevant findings and highlight existing gaps in the literature. Finally, based on these gaps, the conceptual framework for the study is developed, from which the hypotheses, including the moderating role of general AI attitude, are derived (Section \ref{sec:theoretical_foundations_and_hypothesis_development_conceptual_framework_and_hypothesis_development}).

\section{AI in Digital Advertising}\label{sec:theoretical_foundations_and_hypothesis_development_artificial_intelligence_in_digital_advertising}

The field of digital marketing is increasingly permeated by terms such as AI, Machine Learning, and Big Data Analytics. Despite their frequent use, the definitions of these terms are not yet standardized, and there is a lack of clear, universally accepted delineations.

AI is the central concept of this thesis. A universally valid definition remains elusive, in part because the concept of \enquote{intelligence} itself is not precisely settled \cite{bunteKunstlicheIntelligenzZukunft2018, buxmannKuenstlicheIntelligenzMit2021}. For example, the German dictionary Duden \cite{Intelligenz} defines intelligence as \enquote{the ability [of humans] to think abstractly and rationally and to derive purposeful actions from it} [author's translation]. According to Amazon \cite{WhatAIArtificial}, AI is a field of computer science focused on solving cognitive problems normally associated with human intelligence, such as learning, problem-solving, and pattern recognition. A more functional definition considers AI to be a machine employing algorithms or statistical models to carry out tasks associated with the human mind, including perception, cognition, and conversation \cite{longoniResistanceMedicalArtificial2019}. This technology enables the development of self-learning systems that can interpret data to acquire knowledge, which can then be applied to solve new tasks. AI can, for example, respond meaningfully to human conversation, create images and texts, and make decisions based on real-time data inputs. When integrated into a firm, AI can improve business processes, optimize customer experiences, and drive innovation \cite{WhatAIArtificial}.

The concept of AI is not new; it has been in development since the 1950s \cite{bunteKunstlicheIntelligenzZukunft2018}. Its \enquote{birth} is widely attributed to the \enquote{Summer Research Project on Artificial Intelligence} at Dartmouth College in 1956. While this conference established the field, it was followed by a period of stagnation in the 1980s, often referred to as the \enquote{AI winter,} as the technology of the time failed to produce tangible business success \cite{bunteKunstlicheIntelligenzZukunft2018, buxmannKuenstlicheIntelligenzMit2021}.

Today's AI boom is driven by a fundamental shift: the availability of abundant, low-cost computing power and the exponential growth of customer data available for marketing. While the world's largest companies were once primarily in the oil industry, today they are organizations that possess and analyze massive data sets \cite{bunteKunstlicheIntelligenzZukunft2018}. These companies collect customer data, image data, and purchase data, often leveraging sources like user-generated content \cite{WhatAIArtificial}. In this environment, data quality is a primary driver of competitive advantage. AI provides the means to analyze this data faster and more effectively than humanly possible, making the combination of high-quality data and AI a significant competitive tool \cite{bunteKunstlicheIntelligenzZukunft2018}. This growing volume of data, in turn, fuels the development of larger and more capable models, making their outputs increasingly realistic and sophisticated \cite{caoComprehensiveSurveyAIGenerated2023}.

From an economic perspective, AI can be seen as a contributor to the productivity of the classical production factors of labor and capital, or even as an independent production factor in its own right, leading to new growth effects \cite{voepelWieKuenstlicheIntelligenz2018}. However, the full extent of AI's impact on economic growth remains unclear, with different research findings pointing to varied outcomes \cite{buxmannKuenstlicheIntelligenzMit2021}.

According to Bünte, marketing and sales are considered primary beneficiaries of AI, as these departments focus on the often costly interaction with customers. As early as 2018, 80\% of marketing managers recognized the enormous importance of AI for business success \cite{bunteKunstlicheIntelligenzZukunft2018}. In marketing, AI can be used to reduce time expenditure and increase efficiency, particularly in creative endeavors like advertising, which traditionally requires significant human effort \cite{chaisatitkulPowerAIMarketing2024}. By enabling targeted customer engagement, AI can also foster long-term customer loyalty and in rapidly changing markets, AI allows for the cost-effective and rapid modification of products and campaigns \cite{buxmannKuenstlicheIntelligenzMit2021}.

A particularly transformative subset of AI is AI-generated content (AIGC). AIGC utilizes generative AI techniques to create digital content such as images, videos, music, and natural language. In marketing, this is applied to create blog posts, articles, product descriptions, and other materials efficiently and at high quality. \cite{wuAIGeneratedContentAIGC2023}

This capability is primarily powered by Large Language Models (LLMs), such as the one underpinning ChatGPT, which can understand and respond meaningfully to human language \cite{wuAIGeneratedContentAIGC2023, brownLanguageModelsAre2020}. Users provide a prompt, and the system completes the request with a desired output. This process is continually refined through human feedback, which improves the quality of the output and its alignment with user intent. However, these models must be used with caution, as they are trained predominantly on internet data, which can lead to errors and biased information \cite{ouyangTrainingLanguageModels2022}. Simultaneously, generative image models like DALL·E allow users without specialized skills to generate unique images, or modify existing ones, in seconds \cite{wuAIGeneratedContentAIGC2023}. For advertisers, this means that creating a new logo, poster, or campaign visual is no longer a bottleneck.

This shift moves AI from a background tool for data analysis to a visible, active participant in the creation of the advertising message itself. However, for this technology to be effective, its use must be aligned with the brand's values and personality. This alignment is essential for building a foundation of credibility, which in turn has a positive effect on brand perception \cite{marsdenSexLiesAI2019, langeEinflussUnbekannterWerbegesichter2016}.

\section{Conceptualizing Perceived Advertising Credibility}\label{sec:theoretical_foundations_and_hypothesis_development_conceptualizing_perceived_advertising_credibility}

The credibility of advertising campaigns is of great importance for the success of a company. Consumers assess the credibility of advertisements by critically examining both the source and the message of the content. This perceived credibility ultimately affects the attitude toward the brand \cite{hofmannKunstlicheIntelligenzOder2019}.

According to Lange, it is important to create a consistent and stable brand identity and perception. A clear and differentiated idea should be established in the minds of customers. A brand can be viewed from an internal perspective (the brand's self-concept) and an external perspective (perception by external reference groups). The brand identity reflects emotional and symbolic characteristics and consists of the brand's values and personality. Values are fundamental beliefs that the brand represents, while personality includes human characteristics attributed to the brand. In contrast, the brand image is the result of the subjective perception and interpretation of the brand identity by external target groups. The stronger the alignment between the self-image and the external image, the stronger the brand identity. A consistent brand image is crucial for the credibility of and trust in the brand \cite{langeEinflussUnbekannterWerbegesichter2016}. Through this, the brand communicates continuity and individuality against competitors \cite{hofmannKunstlicheIntelligenzOder2019}.

In the context of AI, companies must understand the needs and expectations of their stakeholders to create an unforgettable and credible brand experience. A consistent brand identity and communication must be maintained when integrating AI into advertising to strengthen perceived credibility \cite{hofmannKunstlicheIntelligenzOder2019}.

Credibility has been regarded as an essential factor in the persuasive power of a person and their message since antiquity. In the early 20th century, the concept of credibility was recognized as a scientific discipline in communication research \cite{huschensYouTrustChatGPT2023, appelmanMeasuringMessageCredibility2016}. Research shows that no linear relationship exists between the perceived credibility of a communication source and its persuasive effect. \enquote{Credibility is a perceptual state, i.e. [sic] the outcome of an attribution process in which recipients of messages form judgments about their sources and therefore assess them as credible or not.} \cite{jackobJackob2008Credibility2008} How the credibility of a message is perceived depends on several interlocking factors. The person speaking is just as important as the message itself. Perceived credibility can vary from one recipient to another; one person may perceive it as very credible, while another finds it not credible at all. Therefore, all parts of the communication process must be considered \cite{jackobJackob2008Credibility2008, metzgerCredibility21stCentury2003}.

According to Eisend, a source is perceived as credible if the following aspects are met from the customer's perspective: the company makes valid claims (competence), conveys information conscientiously (trustworthiness), and actively addresses the desires of consumers (dynamism). Eisend describes credibility as the customer's assessment of received information and existing knowledge. \cite{eisendGlaubwuerdigkeitMarketingkommunikation2003}

The credibility of a media message is also influenced by factors unrelated to the source, such as the medium, the transmission channel, and the message itself \cite{metzgerCredibility21stCentury2003}. In general, credibility exists on three different levels: the source level (source credibility), the media level (media credibility), and the message level (message credibility) \cite{appelmanMeasuringMessageCredibility2016}. At the source level, credibility refers to the sender of the information and interpersonal influence. Media credibility concerns the trustworthiness of the communication form and the channel through which the message is sent. Research on message-level credibility focuses on the characteristics and formulations of messages that make them more or less credible \cite{metzgerCredibility21stCentury2003, hellmuellerCredibilityCredibilityMeasures2012}.

\subsection{Source vs. Message Credibility}\label{subsec:theoretical_foundations_and_hypothesis_development_conceptualizing_perceived_advertising_credibility_source_vs_message_credibility}

This study focuses on message-level and source-level credibility. The media level is not considered, as the focus is not on the channels through which the AIGC is communicated.

Message credibility illustrates how the content of a message impacts the perception of credibility, both in relation to the source and the message itself. The two concepts, therefore, overlap. Message factors can often have a greater impact on credibility assessment than source factors. According to the Elaboration Likelihood Model, message factors, in particular, exert greater influence than source characteristics when consumers have high personal relevance and knowledge regarding the message content. This is because such factors increase the motivation to analyze the content of the message. How credible information is perceived thus depends on the recipient's attitudes toward the relevant topics. While disinterested recipients primarily engage with the information superficially, highly involved individuals scrutinize the message more intensively. The former pay more attention to the credibility of the source, while the latter rely on the information in the message and often disregard source characteristics \cite{jackobJackob2008Credibility2008}. In situations where little information about the source is available, recipients must rely even more heavily on message factors to assess credibility \cite{metzgerCredibility21stCentury2003}. Viewers then concentrate on the message level, while the source recedes into the background \cite{jackobJackob2008Credibility2008}. This is particularly relevant to the present study, as participants will not be aware of the advertising's origin.

The significance of source credibility, the second concept considered in this work, was demonstrated as early as 1951 by Hovland \& Weiss. In their study, they had the same message disseminated by a well-known, credible expert and by a non-trustworthy source. The study showed that the credibility of a message depends on the trustworthiness of the source disseminating it \cite{hovlandInfluenceSourceCredibility1951}. Source credibility is understood as a multidimensional construct comprising at least two dimensions: competence and trustworthiness. Competence describes the perceived ability of a source to make valid and reliable statements. Perceived trustworthiness, on the other hand, assesses whether the source intends to communicate correct information. These source-level dimensions can also be expanded by other factors such as security, qualification, and expertise \cite{huschensYouTrustChatGPT2023}.

Institutions can also disseminate persuasive messages, and their perceived credibility can influence consumer attitudes. In marketing literature, the term \enquote{corporate credibility} is used to describe the trustworthiness of institutions. Corporate credibility defines how trustworthy and competent an organization is perceived to be. Organizational credibility shows that the source of a message is not an individual person but a complex institution with an established history. Studies show that the credibility of an organization has a direct influence on consumer attitudes \cite{metzgerCredibility21stCentury2003}. In this thesis, however, the factor of corporate credibility is intentionally excluded to isolate the influence of AI transparency on credibility perception, independent of a specific company's reputation.

\subsection{A Multi-Dimensional Framework for Credibility}\label{subsec:theoretical_foundations_and_hypothesis_development_conceptualizing_perceived_advertising_credibility_a_multi-dimensional_framework_for_credibility}

While the distinction between source and message credibility is foundational, these concepts are themselves multi-dimensional. To create a robust measurement model for the specific context of AIGC, this thesis will adopt a framework based on recent research in this domain. Which, according to Huschens et al. \cite{huschensYouTrustChatGPT2023}, \enquote{was inspired by well-established questionnaires
and measurement approaches in previous studies} \cite{appelmanMeasuringMessageCredibility2016, metzgerCredibility21stCentury2003, hovlandInfluenceSourceCredibility1951, flanaginRoleSiteFeatures2007, winterQuestionCredibilityEffects2014, berloDimensionsEvaluatingAcceptability1969}.

 This model synthesizes the source and message levels into a single, four-dimensional framework for assessing content credibility. These four dimensions are:

\begin{enumerate}

\item \textbf{Competence}: This dimension reflects the perceived expertise and knowledgeability of the source, operationalized through items like \enquote{accurate,} \enquote{complete,} and \enquote{knowledgeable.}
\item \textbf{Trustworthiness}: This dimension captures the perceived honesty and reliability of the source, measured with items such as \enquote{honest,} \enquote{trustworthy,} and \enquote{reliable.}
\item \textbf{Clarity}: This dimension assesses the quality of the message itself, focusing on its comprehensibility. It is measured with items like \enquote{clear,} \enquote{confusing\footnote{Reverse-coded item.},} and \enquote{understandable.}
\item \textbf{Engagement}: This dimension measures the content's ability to capture and hold the reader's attention, using items like \enquote{interesting} and \enquote{maintaining attention.}
\end{enumerate}

This model is particularly advantageous as it cleanly maps onto the foundational concepts discussed in \ref{subsec:theoretical_foundations_and_hypothesis_development_conceptualizing_perceived_advertising_credibility_source_vs_message_credibility}. The \textbf{Competence} and \textbf{Trustworthiness} factors directly measure the classic components of source credibility. The \textbf{Clarity} and \textbf{Engagement} factors, conversely, provide a clear measurement of message credibility \cite{huschensYouTrustChatGPT2023}.

\section{AI Transparency, Disclosure, and Labeling}\label{sec:theoretical_foundations_and_hypothesis_development_ai_transparency_disclosure_and_labeling}

As artificial intelligence transitions from a background tool for data analysis to an active creator of advertising content, its use becomes a salient and potentially critical piece of information for the consumer. This creates a new strategic imperative for marketers regarding transparency. The decision whether and how to disclose the use of AI in creating an advertisement is a novel challenge. This dilemma is intensified by the fact that AIGC is now often perceived as having similar credibility to human-created content \cite{huschensYouTrustChatGPT2023}, placing the burden of disclosure entirely on the company. The following sections \ref{subsec:theoretical_foundations_and_hypothesis_development_ai_transparency_disclosure_and_labeling_defining_ai_transparency_in_advertising} and \ref{subsec:theoretical_foundations_and_hypothesis_development_ai_transparency_disclosure_and_labeling_consumer_response_to_ai-generated_content} will explore the theoretical mechanisms of how such transparency is defined and how consumers are likely to react to it.

\subsection{Defining AI Transparency in Advertising}\label{subsec:theoretical_foundations_and_hypothesis_development_ai_transparency_disclosure_and_labeling_defining_ai_transparency_in_advertising}

In the context of this thesis, AI transparency is defined as the deliberate and overt communication by a firm to inform consumers that a piece of advertising content was generated or significantly assisted by artificial intelligence. This act of disclosure is a specific form of informational \enquote{signal} sent by the company (the \enquote{signaler}) to the consumer (the \enquote{receiver}), who has less information \cite{spenceJobMarketSignaling1973}.

This signal can be interpreted through multiple theoretical lenses. On one hand, transparency can be a core component of a brand's identity, used to build trust and demonstrate a commitment to honesty \cite{langeEinflussUnbekannterWerbegesichter2016, hofmannKunstlicheIntelligenzOder2019}. From this perspective, an AI disclosure label is a signal of forthrightness. This signal could positively influence perceptions of the source's (i.e., the brand's) credibility, specifically its trustworthiness dimension.

On the other hand, the signal also conveys information about the process of the ad's creation. It explicitly states that the creative output is not the product of human endeavor but of an algorithm. This process-related information is what triggers a secondary, more complex set of consumer evaluations, which are explored in the following section \ref{subsec:theoretical_foundations_and_hypothesis_development_ai_transparency_disclosure_and_labeling_consumer_response_to_ai-generated_content}. For the purpose of this study, this transparency is operationalized through a clear disclosure label, which serves as the primary independent variable.

\subsection{Consumer Response to AIGC}\label{subsec:theoretical_foundations_and_hypothesis_development_ai_transparency_disclosure_and_labeling_consumer_response_to_ai-generated_content}

The consumer's response to the knowledge that content is AI-generated is theoretically complex and represents the central tension of this study. The literature suggests two conflicting potential outcomes: a negative reaction to the process and a positive or neutral reaction to the content's quality.

The predominant view in behavioral science is that consumers often exhibit an \enquote{algorithm aversion} \cite{dietvorstAlgorithmAversionPeople2015}. This is a documented cognitive bias where individuals tend to distrust, dislike, or penalize decisions and content made by an algorithm, even when its performance is equal to or superior to that of a human \cite{casteloTaskDependentAlgorithmAversion2019}. This aversion is often rooted in a perception that algorithms lack human intuition, empathy, \enquote{heart,} or genuine creativity. When applied to advertising, this bias would predict a negative outcome. Upon seeing an AI disclosure label, consumers may devalue the advertisement, perceiving it as less authentic, less creative, or less reliable. This would lead to a direct decrease in perceived credibility, particularly on the trustworthiness and engagement dimensions.

However, this negative bias is not the only possible outcome. Recent research comparing human- and AIGC has introduced a significant nuance. A study by Huschens et al. \cite{huschensYouTrustChatGPT2023} found that while participants did not rate AI-generated texts differently on competence or trustworthiness, they surprisingly rated them as clearer and more engaging than the human-written equivalents. This finding suggests that consumers may, in some contexts, perceive the output of AI as being of higher quality on certain message-level attributes (like clarity), even if they are averse to the process.

This study is therefore positioned to investigate this critical trade-off: What is the net effect on overall perceived credibility when a consumer is explicitly told that an advertisement was created by AI?

\section{Synthesis of Current Research}\label{sec:theoretical_foundations_and_hypothesis_development_synthesis_of_current_research}

The research on the perception and credibility of AIGC provides varying and sometimes contradictory results.

For instance, Chaisatitkul et al. \cite{chaisatitkulPowerAIMarketing2024} examined attitudes and perceptions towards work created by generative AI compared to human-generated content. They analyzed ad scripts from ChatGPT and storyboards from DALL·E, surveying three groups: end-users, marketing professionals, and agency employees. The findings showed that end-users (Group 1) held a positive attitude towards the AIGC; interest in the script even increased slightly after its AI origin was revealed. The professional groups (2 and 3) also showed positive attitudes, particularly regarding the visual content, practicality, and brand trust. Notably, all groups preferred the first script and storyboard, which they knew was AI-generated, over the second, human-created one.

In contrast, other studies underscore the negative effects of AI. Haupt et al. \cite{hauptConsumerResponsesHumanAI2025} investigated the potential of human-AI collaboration in creating advertising copy, examining the impact of authorship on message credibility and brand attitude. Their study analyzed texts authored by: (1) a human, (2) a human with AI support, (3) an AI under human control, or (4) an AI alone. The results showed that AI-based texts are evaluated similarly to human-created texts as long as their authorship is concealed. However, with transparent AI use, consumer reactions varied. Ad copy created by an AI or by a human with AI support was perceived as less credible and had negative effects on brand attitude. Conversely, content written by an AI under human control had no significant negative impact. This suggests that negative perceptions—a phenomenon known as algorithm aversion—can be mitigated if human control is emphasized.

The study by Hofmann \cite{hofmannKunstlicheIntelligenzOder2019} examined how the use of AI algorithms in brand management affects brand credibility, focusing on the streaming industry, particularly Netflix. Netflix is transparent about its use of AI in data analysis and product management, a deliberate strategy to position itself as an innovator. The findings showed that the discussion of AI in brand management is relevant to the perception of brand credibility. The study concluded that AI integration must not only be technologically sensible but also align with the established values and personality of the brand to be effective.

Adding nuance, Huschens et al. \cite{huschensYouTrustChatGPT2023} investigated the perceived credibility of content created by humans versus content generated by LLMs like ChatGPT. The results showed that participants judged the credibility of both human and AI-created content to be roughly the same. Furthermore, no differences were reported in the perceptions of competence or trustworthiness. Notably, the AIGC was even rated as clearer and more engaging than the human-generated content.

Finally, research by Marsden \cite{marsdenSexLiesAI2019} argues for the importance of transparent communication, suggesting that companies should clearly explain how AI is used to reduce skepticism and maintain trust. Marsden emphasizes that a company's persuasive power is a decisive factor in mitigating mistrust towards AI and fostering a positive attitude. Because credibility plays a crucial role in brand assessment, companies must understand their stakeholders' expectations regarding AI to create an authentic brand experience.

The existing literature, therefore, presents a mixed and incomplete picture. Although research on AI and marketing is increasing, there are still significant gaps concerning advertising, credibility, and AIGC. Conflicting evidence exists, and given the novelty of the topic for the general public, consumer perceptions are still in flux.

Several specific gaps emerge from this body of research, which the present thesis aims to address. First, while studies like Huschens et al. \cite{huschensYouTrustChatGPT2023} compare AI vs. human content, the critical gap is the effect of disclosure itself. This study will, therefore, test the difference in credibility perceptions when the same advertisement is presented with and without an AI disclosure label. Second, much of the research has focused on text-based content (e.g., ad copy, news articles). This thesis will address a gap by focusing on the impact of disclosure on visual advertisements. Finally, this study will focus specifically on the end-consumer, providing clear managerial implications for how AI disclosure strategies directly impact brand perception.

\section{Conceptual Framework and Hypothesis Development}\label{sec:theoretical_foundations_and_hypothesis_development_conceptual_framework_and_hypothesis_development}

Based on the theoretical foundations and the research gaps identified in the synthesis of current literature, this thesis proposes a moderated-mediation model. The central research question is:

\textit{To what extent does the transparent disclosure of AI-generated content in advertising impact its perceived credibility?}

The conceptual framework posits that an AI disclosure label (independent variable) will have a direct effect on the perceived credibility of an advertisement (dependent variable).

This study will adopt the multi-dimensional credibility model from section \ref{subsec:theoretical_foundations_and_hypothesis_development_conceptualizing_perceived_advertising_credibility_a_multi-dimensional_framework_for_credibility}, which breaks perceived credibility into four distinct dimensions: competence, trustworthiness, clarity, and engagement \cite{huschensYouTrustChatGPT2023}. The hypotheses will therefore test the effect of the AI label on each of these four dimensions individually.

Furthermore, this relationship is not assumed to be uniform across all consumers. As credibility is a subjective \enquote{perceptual state} that varies by recipient \cite{jackobJackob2008Credibility2008, metzgerCredibility21stCentury2003}, this framework introduces a key moderator: the consumer's general attitude toward AI. This pre-existing attitude is expected to influence the strength or direction of the consumer's reaction to the AI disclosure label.

The following sections \ref{subsec:theoretical_foundations_and_hypothesis_development_conceptual_framework_and_hypothesis_development_the_effect_of_ai_transparency_on_perceived_credibility} and \ref{subsec:theoretical_foundations_and_hypothesis_development_conceptual_framework_and_hypothesis_development_the_moderating_effect_of_general_ai_attitude} will now formally derive the testable hypotheses from this framework \cite{hartmannHypothesenTestenEinfuehrung2015, raithelQuantitativeForschung2006}.

\subsection{The Effect of AI Transparency on Perceived Credibility}\label{subsec:theoretical_foundations_and_hypothesis_development_conceptual_framework_and_hypothesis_development_the_effect_of_ai_transparency_on_perceived_credibility}

As established in section \ref{subsec:theoretical_foundations_and_hypothesis_development_ai_transparency_disclosure_and_labeling_consumer_response_to_ai-generated_content}, the literature presents a central conflict. On one hand, consumers exhibit a well-documented \enquote{algorithm aversion} \cite{dietvorstAlgorithmAversionPeople2015}, which suggests a negative reaction to content known to be non-human \cite{casteloTaskDependentAlgorithmAversion2019, hauptConsumerResponsesHumanAI2025}. This bias would predict that labeling an ad as AI-generated will harm its credibility. On the other hand, research has also shown that AI-generated content can be perceived as equal to or even better than human content on specific message-level attributes like \enquote{clarity} and \enquote{engagement} \cite{huschensYouTrustChatGPT2023}.

The experimental design of this thesis is built to test this exact question. It will compare consumer reactions to advertisements that include explicit AI disclosure labels against a control-group version of the same advertisement that has no label at all. This design isolates the specific, real-world impact of adding a transparency label to an otherwise standard advertisement.

The following hypotheses are posited, predicting that the negative effects of algorithm aversion will outweigh any potential perceived benefits of the content's quality, leading to an overall negative impact on credibility.

Based on the four-dimensional credibility model \cite{huschensYouTrustChatGPT2023}, this leads to the following four hypotheses:

\begin{itemize}
\item \textbf{H1}: The presence of an AI disclosure label will cause consumers to evaluate an advertisement's competence (accurate, complete, knowledgeable) more negatively than the same advertisement presented without a label.
\item \textbf{H2}: The presence of an AI disclosure label will cause consumers to evaluate an advertisement's trustworthiness (honest, trustworthy, reliable) more negatively than the same advertisement presented without a label.
\item \textbf{H3}: The presence of an AI disclosure label will cause consumers to evaluate an advertisement's clarity (clear, confusing, understandable) more negatively than the same advertisement presented without a label.
\item \textbf{H4}: The presence of an AI disclosure label will cause consumers to evaluate an advertisement's engagement (interesting, maintaining attention) more negatively than the same advertisement presented without a label.
\end{itemize}

\subsection{The Moderating Effect of General AI Attitude}\label{subsec:theoretical_foundations_and_hypothesis_development_conceptual_framework_and_hypothesis_development_the_moderating_effect_of_general_ai_attitude}

As noted, perceived credibility is subjective and can vary significantly depending on the individual recipient \cite{jackobJackob2008Credibility2008, metzgerCredibility21stCentury2003}. In the context of AI, one of the most salient individual differences is a person's pre-existing disposition or \enquote{general attitude.}

It is logical to assume that the negative impact predicted in H1-H4 will not be universal. Consumers who already have a positive attitude toward AI (e.g., they find it useful, innovative, or exciting) may not experience algorithm aversion. For them, an \enquote{AI-created} label might be a neutral or even positive signal. Conversely, consumers who hold a negative attitude (e.g., they find AI threatening, inauthentic, or untrustworthy) will likely have their biases confirmed by the label, leading to a much stronger negative evaluation.

Therefore, the general attitude toward AI is hypothesized to act as a moderator, influencing the strength of the relationship between the AI label and the perceived credibility dimensions.

\begin{itemize}
\item \textbf{H5}: The consumer's general attitude toward AI has an influence on the perceived competence of the advertisement.
\item \textbf{H6}: The consumer's general attitude toward AI has an influence on the perceived trustworthiness of the advertisement.
\item \textbf{H7}: The consumer's general attitude toward AI has an influence on the perceived clarity of the advertisement.
\item \textbf{H8}: The consumer's general attitude toward AI has an influence on the perceived engagement of the advertisement.
\end{itemize}



\chapter{Research Methodology}\label{ch:research_methodology} % 6-8 Pages

\section{Experimental Design}\label{sec:research_methodology_experimental_design}

\section{Stimulus Material Development and Pre-testing}\label{sec:research_methodology_stimulus_material_development_and_pre-testing}

\section{Sampling Strategy and Data Collection Procedure}\label{sec:research_methodology_sampling_strategy_and_data_collection_procedure}

\section{Measurement Instruments}\label{sec:research_methodology_measurement_instruments}

\subsection{Independent Variable: AI Transparency Manipulation}\label{subsec:research_methodology_measurement_instruments_independent_variable_ai_transparency_manipulation}

\subsection{Dependent Variable: Perceived Credibility}\label{subsec:research_methodology_measurement_instruments_dependent_variable_perceived_credibility}

\subsection{Moderating Variable: General AI Attitude}\label{subsec:research_methodology_measurement_instruments_moderating_variable_general_ai_attitude}

\section{Data Analysis Strategy}\label{sec:research_methodology_data_analysis_strategy}



\chapter{Results}\label{ch:results} % 8-10 Pages

\section{Sample Characteristics and Descriptive Statistics}\label{sec:results_sample_characteristics_and_descriptive_statistics}

\subsection{Sociodemographic Profile}\label{subsec:results_sample_characteristics_and_descriptive_statistics_sociodemographic_profile}

\subsection{Descriptive Statistics for Key Variables}\label{subsec:results_sample_characteristics_and_descriptive_statistics_descriptive_statistics_for_key_variables}

\section{Manipulation and Confound Checks}\label{sec:results_manipulation_and_confound_checks}

\section{Hypothesis Testing}\label{sec:results_hypothesis_testing}

\section{Exploratory Analyses}\label{sec:results_exploratory_analyses}



\chapter{Discussion}\label{ch:discussion} % 4-6 Pages

\section{Summary and Interpretation of Findings}\label{sec:discussion_summary_and_interpretation_of_findings}

\section{Theoretical Implications}\label{sec:discussion_theoretical_implications}

\section{Managerial and Practical Implications}\label{sec:discussion_managerial_and_practical_implications}



\chapter{Conclusion}\label{ch:conclusion} % 2-3 Pages

\begin{figure}[H]
  \centering
  \includegraphics[width=0.3\textwidth]{barcode_default.png}
  \caption{User-Flow-Diagramm des tollen Algorithmus.}
  \label{fig:user-flow-diagramm}
\end{figure}

\section{Concluding Summary}\label{sec:conclusion_concluding_summary}

\section{Limitations and Future Research Directions}\label{sec:conclusion_limitations_and_future_research_directions}

\backmatter

%%%%%%%%%%%%%%%%%%%
%% Appendix (1) or Appendices (>1)
%%%%%%%%%%%%%%%%%%%

\appendix
\addchap{Appendix}

%\addsec{Anhang A: Eine tolle Tabelle.}
%\label{appendix:antworttypen}
%\begin{table}[H]
%\centering
%\begin{tabular}{|l|p{9cm}|}
%\hline
%\textbf{Headline 1} & \textbf{Headline 2} \\
%\hline
%Titel 1 & Beschreibung 1 \\
%\hline
%Titel 2 & Beschreibung 2 \\
%\hline
%\end{tabular}
%\caption{Übersicht der tollen Antworten.}
%\end{table}

\cleardoublepage
\phantomsection

%%%%%%%%%%%%%%%%%%%
%% References
%%%%%%%%%%%%%%%%%%%

\addcontentsline{toc}{chapter}{References}
\printbibliography[title=References]

%%%%%%%%%%%%%%%%%%%
%% Formalia (German?)
%%%%%%%%%%%%%%%%%%%

\addchap{Erklärung zur Verwendung von KI-Systemen}
Als Hilfsmittel zur Erstellung der vorliegenden Bacheloararbeit wurde generative KI verwendet. Die Nutzung dieser generativen KI diente der Unterstützung bei der Recherche und Ideenfindung sowie deren Formulierung. Alle wissenschaftlichen Analysen, Interpretationen und Schlussfolgerungen basieren auf eigener Arbeit und wurden lediglich durch den kritischen Umgang mit den von der KI generierten Vorschlägen ergänzt.
\vspace{20pt}
\begin{flushright}
$\overline{~~~~~~~~~~~~~~~~~\mbox{\ShowBaAuthor; \today}~~~~~~~~~~~~~~~~~}$
\end{flushright}

\addchap{Zustimmung zur Plagiatsüberprüfung}
Hiermit willige ich ein, dass zum Zwecke der Überprüfung auf Plagiate meine vorgelegte Arbeit in digitaler Form an PlagScan (www.plagscan.com) übermittelt und diese vorübergehend (max. 5~Jahre) in der von PlagScan geführten Datenbank gespeichert wird sowie persönliche Daten, die Teil dieser Arbeit sind, dort hinterlegt werden.

\begin{small}
Die Einwilligung ist freiwillig. Ohne diese Einwilligung kann unter Entfernung aller persönlichen Angaben und Wahrung der urheberrechtlichen Vorgaben die Plagiatsüberprüfung nicht verhindert werden. Die Einwilligung zur Speicherung und Verwendung der persönlichen Daten kann jederzeit durch Erklärung gegenüber der Fakultät widerrufen werden.
\end{small}
\vspace{20pt}
\begin{flushright}
$\overline{~~~~~~~~~~~~~~~~~\mbox{\ShowBaAuthor; \today}~~~~~~~~~~~~~~~~~}$
\end{flushright}

\addchap{Eidesstattliche Erklärung}
Hiermit versichere ich, dass ich die vorgelegte Bachelorarbeit selbstständig verfasst und noch nicht anderweitig zu Prüfungszwecken vorgelegt habe. Alle benutzten Quellen und Hilfsmittel sind angegeben, wörtliche und sinngemäße Zitate wurden als solche gekennzeichnet.
\vspace{20pt}
\begin{flushright}
$\overline{~~~~~~~~~~~~~~~~~\mbox{\ShowBaAuthor; \today}~~~~~~~~~~~~~~~~~}$
\end{flushright}

\end{document}
